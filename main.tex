\documentclass[notitlepage, draft]{article}
\usepackage[utf8]{inputenc}
\usepackage{geometry}
\usepackage{graphicx}
\usepackage{float}
\usepackage[most]{tcolorbox}
\usepackage{framed}
\usepackage{makeidx}
\usepackage[brazil]{babel}
%\pagestyle{empty}
%Define color serve para definir uma cor que não esteja já adicionada à um dos pacotes utilizados.Esse pode ser HTML, RGB(0-255), cmyk, gray e rgb(0-1).
\definecolor{textGray}{RGB}{230, 230, 230}
\definecolor{backGray}{RGB}{252, 252, 252}
\pagecolor{backGray}


\usepackage{geometry} %Formatação da página.
 \geometry{
 a4paper,
 total={170mm,257mm},
 left=20mm,
 top=20mm,
 }

\tcbset{ 
    frame code={}
    center title,
    left=0pt,
    right=0pt,
    top=0pt,
    bottom=0pt,
    colback= textGray,
    %colback define a cor que será utilizada por trás do texto quando chamado o comando \begin{tcolorbox} // \end{tcolorbox}
    colframe=white,
    width=\dimexpr\textwidth\relax,
    enlarge left by=0mm,
    boxsep=5pt,
    arc=0pt,outer arc=0pt,
    }%Esse comando define a formatação do campo dentro do tcolorbox. 
    % tcbset = T Color Box Set.
    
    


\title{\textbf{SGLZ} - Sistema de Gerenciamento Lojão do Zé}


\begin{document}
    \date{}
    \maketitle

\vspace{15cm}

\begin{tabular}{|p{10.5 cm}|p{4.5 cm}|}
    \hline
    \multicolumn{2}{|c|}{Grupo} \\
    \hline
    Nome Completo & Matrícula \\
    \hline
    Deborah Condé  & 201465629AC \\
    Bruno Ferreira Cangussu  & 201565014AC \\
    Osiel do Couto  & 201865192A \\
    Rômulo Soares  & 201665219AC \\
    (NOME)  & (MATRÍCULA) \\
    \hline
\end{tabular}
    
    
    \newpage
    
    
    \tableofcontents
    % \tableofcontents faz o sumário do documento.
    


    
    \newpage %Pula para a próxima página 
    
    \section{Introdução}
        \subsection{Propósito}
    O propósito do Documento de Especificação de Requisitos é delinear os requisitos do software a ser construído, descrevendo suas funcionalidades e características. O público alvo do documento são clientes, gerentes e desenvolvedores do projeto.
    
        \subsection{Escopo}
    O Sistema de Gerenciamento Lojão do Zé auxiliará o processo de gerenciamento de clientes, vendas, estoque e funcionários da empresa, visando sua agilização, bem como a facilitação do acesso aos dados e aumento de sua confiabilidade.
    
        \subsection{Definições e Abreviações}
        
            \begin{itemize}
            
                \item \textbf{SGLZ:} nome dado ao Sistema de Gerenciamento Lojão do Zé;
         
                \item \textbf{RF}: requisito funcional;
     
                \item \textbf{RNF}: requisito não funcional;
     
                \item \textbf{Administrador}: usuário do sistema com acesso às ferramentas de gerenciamento de produtos, estoque e funcionários, além das ferramentas disponíveis aos vendedores. Tem acesso total ao sistema;
     
                \item \textbf{Usuário do sistema}: usuário do sistema, seja ele administrador ou vendedor;
    
            \end{itemize} 
    
        \subsection{Visão Geral do Documento}
        
            \begin{itemize}
            
                \item \textbf{Seção 2  - Descrição Geral:} apresenta uma visão geral do sistema, especificando a perspectiva do produto e detalhamento do escopo do sistema através da discretização das funções do produto. Além disso, são explicitadas as características gerais dos usuários do produto e as restrições que poderão limitar as possibilidades de desenvolvimento.
        
                \item \textbf{Seção 3  - Descrição dos Requisitos Funcionais (RF):} apresentação de todos os requisitos funcionais do sistema. Descreve as principais ações do produto, considerando a aceitação e processamento das entradas e o processamento e geração das saídas.
        
                \item \textbf{Seção 4  - Requisitos Não Funcionais:} apresentação de todos os requisitos não funcionais do sistema. Descreve todos os aspectos qualitativos do sistema, explicitando os detalhes de facilidade de uso, manutenibilidade, confiabilidade, desempenho, segurança, distribuição, adequação a padrões e requisitos de hardware e software.
                
            \end{itemize}
    
    \newpage
    
    \section{Descrição Geral}
    
        \subsection{Perpectivas do Produtos}
             O sistema é desenhado para ser executado em servidor Web remoto. Para que o usuário acesse o sistema, é necessário ter um computador ou dispositivo móvel com acesso à internet e a um navegador (ex: Chrome, Firefox, Internet Explorer etc.). A interação com o sistema se dará por interface gráfica.
    
        \subsection{Funções do Produto}
            %.\begin{itemize}  \item  \end{itemize} cria uma listagem de itens.
            \begin{itemize}
                    \item Função Exemplo;
            \end{itemize}
    
        \subsection{Restrições}
            O sistema deve ser desenvolvido com os recursos disponíveis na plataforma Web.
    
    \newpage
    
    \section{Descrição dos Requisitos Funcionais (RF)}
    
        \begin{table}[H] %O H faz com que a tabela se posicione de acordo com a posição dela no código. Esse tipo de alinhamento é encontrado dentro do pacote: \usepackge{float}
        %O comando  \begin{tcolorbox} // \end{tcolorbox} faz a coloração por trás do texto.
        
            \begin{tcolorbox}
                \textbf{RF001 – Requisito Funcional de Exemplo}
            \end{tcolorbox}
    
            \quad Descrição do Requisito Funcional de Exemplo.
            %\quad cria um pequeno espaço antes de começar o texto.
        \end{table}
        
        \begin{table}[H]
        
            \begin{tcolorbox}
                \textbf{RF002 – Cadastrar Usuário}
            \end{tcolorbox}


            \quad O sistema deve cadastrar um usuário com os dados: \textbf{Id, senha, nome, e-mail, endereço (cidade, estado e cep), CPF, telefone, Complemento} e \textbf{tipo}.
        \end{table}
 
    \newpage
    
    \section{Descrição dos Requisitos Não Funcionais (RNF)}
    
        \begin{table}[H] 
        
            \begin{tcolorbox}
                \textbf{RNF001 – Requisito não Funcional de Exemplo}
            \end{tcolorbox}
        
            \quad Descrição do Requisito não Funcional de Exemplo.
            
        \end{table}
        
        \begin{table}[H]
            \begin{tcolorbox}
                \textbf{RNF002 – Tempo de Resposta das listagens}
            \end{tcolorbox}

            \quad O tempo de resposta das listagens não deve ser superior a dois segundos.

        \end{table}

\end{document}